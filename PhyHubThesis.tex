\documentclass{ctexart} 
% \documentclass{article} 如果用英文
% \documentclass[10pt,a4paper]{ctexart}  字体大小和纸张大小,默认分别为10pt和a4paper
% 五号 = 10.5pt,小四=12pt,四号=14pt
% 其他可选参量如twocolumn, 两行排版

\usepackage{biblatex} %[style=gb7714-2015]{biblatex} 可以选择样式
\addbibresource{PhyHubThesis.bib} % 把这里改成实际的文件名

% 令参考资料能够加入目录中:
\defbibheading{bibliography}[\bibname]{% 
	\section{#1}% 
	\markboth{#1}{#1}}

% 将路径绝对路径 (Windows) 或相对路径 (Mac OS或Linux)
% 使用「/」而不是「\」
\newcommand{\PATH}{路径/}

% 引用的宏包:
% 宏包的使用, 可以在命令行运行texdoc <宏包名>获得文档
\usepackage{multicol} % 分栏 (全局分栏建议在文档类处设置)
\usepackage{amsmath} % AMS数学标准
\usepackage{amssymb} % 数学符号
\usepackage{mathrsfs} % 花体
\usepackage{esint} % 积分
\usepackage{siunitx} % 标准SI数值和单位处理
\usepackage{tikz} % 绘图
\usepackage{float} % 浮动体 (供图片, 表格等) 扩展, 主要用于提供h模式
\usepackage{graphicx} % 插入图片
\usepackage[font=small, skip=5pt]{caption} % 缩小题注字体和题注与图片距离
\usepackage{subcaption} % 子图和子图的题注
\usepackage{svg} % svg位图
\usepackage{wrapfig} % 简单的图文绕排
\usepackage[inline]{enumitem} % 编号
\usepackage{geometry} % 调整页边距
% \geometry{left=1.6cm,right=1.6cm}
\usepackage{xcolor} % 颜色
\usepackage[colorlinks=true,bookmarks=true]{hyperref} % 引用, 交叉引用, 图表等的链接; 生成书签
\hypersetup{linkcolor=[rgb]{1,0.27,0},bookmarksopen = true}% 更多设置请查阅: texdoc hyperref


% 定义一些笔者常用的指令:
\newcommand{\me}{\mathrm{e}} % 自然对数的底
\newcommand{\mi}{\mathrm{i}} % 虚数单位
\newcommand{\md}{\mathrm{d}} % 微分算子d
\newcommand*{\basis}[1]{\hat{\boldsymbol{#1}}} % 基底
\newcommand*{\bv}{\boldsymbol} % 向量加粗
\newcommand*{\pdiff}[2]{\frac{\partial #1}{\partial #2}} % 导数 \diff{y}{x}
\newcommand*{\npdiff}[3]{\frac{\partial^{#1} #2}{\partial #3^{#1}}} % n阶导数 \ndiff{n}{y}{x}
\newcommand*{\diff}[2]{\frac{\mathrm{d} #1}{\mathrm{d} #2}} % 偏导数\pdiff{y}{x}
\newcommand*{\ndiff}[3]{\frac{\mathrm{d}^{#1} #2}{\mathrm{d} #3^{#1}}} % n阶偏导数\pdiff{n}{y}{x}

% 笔者习惯的运算符:
\DeclareMathOperator{\tg}{tg}
\DeclareMathOperator{\ctg}{ctg}
\DeclareMathOperator{\arctg}{arctg}
\DeclareMathOperator{\sh}{sh}
\DeclareMathOperator{\ch}{ch}
% \DeclareMathOperator*{\指令}{显示} 
% 带星号的版本会像\lim一样

% 文章标题页信息:
\title{标题}
\author{ 作者1\thanks{作者1的联系方式}
	\and
		作者2\thanks{作者2的联系方式}}
\date{\today} % 自动生成日期

\begin{document}
\maketitle % 打印标题

\begin{abstract}
	摘要
	\centering % 使得关键字居中
	\textbf{关键字:}
		关键字1\ 
		关键字2\  
		关键字3
\end{abstract}

\tableofcontents
\newpage

\section{模板使用说明}
\subsection{编译}
把导言区的\verb|\newcommand{\PATH}{路径}|改成此文档的相对路径 (MAC OS或Linux) 或绝对路径 (Windows), 然后在该目录下使用命令行依次运行:\\[1pt]
\texttt{xelatex} 文件名 (可以省略.tex)\\
\texttt{biber} 文件名 (可以省略.tex)\\
\texttt{xelatex} 文件名\\
\texttt{xelatex} 文件名\\

\subsection{使用说明}

引用.bib中的书目: \cite{Knuth}.

\begin{figure}[h] %
\centering
\begin{subfigure}{.5\textwidth}
	\centering
	\includegraphics[width=6cm]{\PATH latex-project-logo.pdf}
	\caption{题注}
	\label{子图1}
\end{subfigure}%
\begin{subfigure}{.5\textwidth}
	\centering
	\includegraphics[width=6cm]{\PATH latex-project-logo.pdf}
	\caption{题注}
	\label{子图2}
\end{subfigure}
\caption{一个并排放置图片的示例的示例}
\end{figure}

交叉引用的例子: 图~\ref{子图1} 是LaTeX Project的图标.

单行公式:
\begin{equation}\label{公式的标签}
	\int \diff y x\,\md x = y(x) + C.
\end{equation}

多行公式与对齐:
\begin{align}
	\nabla\cdot \bv D &= \rho
	\\
	\nabla \times \bv E & =  - \pdiff{\bv B}{t}
	\\
	\nabla \cdot \bv B &= 0
	\\
	\nabla \times \bv H &= \bv j + \pdiff{\bv D}{t}
\end{align}

\section{\texorpdfstring{在目录和文档中显示的标题}{在书签中显示的标题}}

使用\verb|\texorpdfstring{在目录和文档中显示的标题}{在书签中显示的标题}|来处理标题上有\LaTeX{}公式时标签不能正常显示的问题.

\nocite{*} % 这个表示列出所有没有在文中被引用的参考文献
\printbibliography[heading=bibnumbered, title={参考文献}]
\end{document}